\chapter{Introduction}

\section{Project Motivations}

As the internet became available to more people across the world, access to digital information became a fundamental principle of democratic engagement and global communication. The concept of humans being able to know exactly what is happening at any time on any part of the globe is very new, and some countries and regimes have pushed back on this idea. This pushback has led to the censorship of information on the internet in some parts of the world. While countries such as China and Iran are known for having extensive censorship mechanisms in place, this work is motivated by the need to examine different types of regimes - those being Iraq and Ireland, and the comparison of the two.

Ireland, a member of the European Union, uses a censorship model based on limited and specific censorship. Most of Ireland's censorship is rooted in judicial and legal oversight. By contrast, Iraq demonstrates a relatively more reactive approach to censorship, characterized by intermittent shutdowns and varying regional enforcement. These two countries have large differences in government, culture, and social norms, which offers a unique perspective on how different regimes manage internet freedom.

This project is also motivated by the need for more publicly available empirical data in regards to internet interference. The Open Observatory of Netowrk Interference (OONI) Probe allows for real data to be published to further contribute to the analysis of this work. 

\section{Project Goals}

The aim of this project is to conduct a comparative analysis of internet censorship practices in Ireland and Iraq. By using tools such as the Open Observatory of Network Interference (OONI) Probe, this work aims to identify and document the presence, mechanisms, and the extent of internet censorship in both countries.

\section{Internet Censorship and Privacy}

\subsection{Overt vs. Covert Censorship}

Censorship can be implemented in many different ways, but there are two main categories: Overt and Covert Censorship. Overt censorship is openly implemented by governments, ISP's, or legal courts to block or restrict access to certain types of content, or specific websites. When the content a user is trying to access is blocked using overt censorship, it is made very clear to the user that it is blocked. An example of this is the 'Golden Shield Project', which is China's internet censorship project. This project blocks access to websites such as google and facebook, and the citizens of China are often aware that they websites have been blocked by the government \cite{standfordGreatFirewallOfChina}.

Covert censorship is more often harder to detect. Search engine manipulation, throttling or slowness, and shadow banning are some of the primary methods of covert censorship. The goal of this type is to make censorship more difficult to detect by users, and is often disguised as technical issues. 

\subsection{Privacy}

User Privacy across the internet is directly tied to censorship efforts from different regimes. Censorship often involves the state or corporate monitoring of internet users, and governments that impose censorship frequently justify it using security concerns while often violating privacy rights in the process. In countries where censorship is highly enforced, using anonymity tools to circumvent censorship can protect the right to free expression and access to information. For instance, the \textit{Human Rights Watch} advises people in China to make use of the Tor Browser to avoid abuses by the state \cite{Privacy2017}. 

Based on a meta-analysis of studies related to internet privacy concerns, privacy literacy, and the adoption of privacy-protective measures, it was found that there is no strong correlation between national privacy laws and protective behaviors \cite{baruh2017online}. This suggests that individuals do not rely on legal protections in their country, and more often take privacy into their own hands. It was also found that culture did not impact the use of privacy-protective behaviors in different countries. 

While it may be easy to think censorship is only prevalent in non-western countries, such as China or Russia, it can also happen in democratic states. Weak privacy protections can lead to surveillance capitalism, where companies act as de facto censors by shaping information flows based on user data \cite{schwartz1999internet}. For example, during the COVID-19 pandemic in the United States, it was recently revealed that Meta (formerly Facebook) was asked to censor certain information regarding COVID-19 \cite{pbsZuckerbergSays}. The United States Government and Meta actively engaged in the censorship of the people's right to free speech and expression, as humor and satire was also removed from the platform.

\subsection{Global Censorship (Chris)}
\label{sec:Chris-Global-Censorship}

Experts suggest that censorship on the internet is increasing at an alarming rate. “The majority of countries that censor content do so across all four themes, although the depth of the filtering varies. The study confirms that 40 percent of these 2,046 websites can only be reached by an 
encrypted connection (denoted by the "HTTPS" prefix on a web page, a voluntary upgrade from "HTTP")” \cite{zittrain2017shifting}. It is also clear that more and more countries are viewing this as a necessary solution to the unique problems they have. Whether this is appropriate or not, it is happening, and users should be aware of this. 

Governments have a vested interest in maintaining control over telecommunications industries and public internet use. Whether protecting state secrets, preventing cyber crime piracy or acts of terrorism, insulating from perceived negative influence, aiding in the creation of propaganda or otherwise; a large majority of governments choose to exercise inordinate control over the 
information available to its public.  

As more governments and entities began to engage in this, it became increasingly important to hold them accountable. As a result, the ‘Enemies of the Internet’ list was devised. It contains the governments and entities that actively engage in the repression of online freedoms, in the form of censorship and surveillance. As of 2014, there were 19 governments that fit this criterion but by now this number has likely increased \cite{GlobalCensorshipRef2}. Traditionally, censorship involved monitoring a handful of media and cutting undesirable content, potentially replacing this with a message more in line with the agenda and norms of the locale. However, with the advent of the internet, this distribution of information became decentralised and thus allowed for more expression and freedom in the content consumed by a user. As a result, censorship has become more difficult to conduct, but potentially easier to get away with. Nowadays, governments leverage points of control, network-level filtering and many other techniques to block undesirable content.




