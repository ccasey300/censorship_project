\chapter{Conclusions}

This work set out to research and compare internet censorship practices in Ireland and Iraq. These two nations have significantly different governments and cultures. By using both direct network testing using the OONI Probe and published OONI data, this work has identified the distinct methods and motivations behind censorship in each country. 

As a member of the European Union, Ireland implements a limited and legally rooted censorship model. Ireland mainly blocks illegal and pirated content, which is implemented through the country's legal system or EU compliance. The OONI data shows little evidence of widespread censorship of non-illegal content, no use of advanced censorship methods like TLS / SNI-based filtering, and no efforts to block circumvention tools.

Iraq implements a more decentralized and reactionary censorship model, blocking a wider range of content than Ireland. The Iraqi government is known to shut off internet access during unrest or national exams in parts of the country. The OONI data shows evidence of TLS/ SNI-based filters in specific parts of the country, but no evidence supports the widespread implementation of these advanced mechanisms. Additionally, there is some evidence that circumvention tools, such as Psiphon, are being blocked in some areas, but there is nothing to support widespread efforts to block these tools. These findings point to regional and situational censorship, driven by political or cultural events rather than being rooted in a legal basis.

This comparative analysis contributes to the growing importance of digital rights and government accountability using empirical data and structured analysis. The importance of projects like OONI is that they allow people to learn about the presence and nature of internet censorship in their own countries and allow researchers to identify global trends and document network interference. As the internet continues to grow in importance to civil discourse, education, and access to information, vigilance against censorship becomes foundational.