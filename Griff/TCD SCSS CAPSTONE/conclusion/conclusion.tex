\chapter{Conclusions}

This work set out to research and compare internet censorship practices in Ireland and Iraq. These two nations have significantly different governments and cultures. By using both direct network testing using the OONI Probe and published OONI data, this work has identified the distinct methods and motivations behind censorship in each country. 

Ireland as a member of the European Union implements a limited and legally rooted censorship model. Ireland mainly blocks illegal and pirated content, and these blocks are implemented through the country's legal system or through EU compliance. The OONI data shows that there is little evidence of widespread censorship of non-illegal content, no use of advanced censorship methods like TLS / SNI-based filtering, and no efforts to block circumvention tools.

Iraq implements a more decentralized and reactionary censorship model, where a wider range of content is blocked when compared to Ireland. The Iraqi government is known to shut off internet access in parts of the country during times of unrest or national exams. The OONI data shows that there is evidence TLS/ SNI-based filters in specific parts of the country, but there is no evidence to support that there is widespread implementation of these advanced mechanisms. Additionally there is some evidence of some circumvention tools, such as Psiphon, being blocked in some areas, but there is nothing to support widespread efforts to block these tools. These findings point to regional and situational censorship, driven by political or cultural events rather than being rooted in a legal basis.

This comparative analysis contributes to the growing importance of digital rights and government accountability by using empirical data and structured analysis. The importance of projects like OONI allows for people to learn about the presence and nature of internet censorship in their own countries, and gives researchers the ability to identify global trends and document network interference. As the internet continues to grow in importance to civil discourse, education, and access to information, vigilance against censorship becomes foundational.