\chapter{Methodology}

\section{The OONI Probe}

\subsection{Background of OONI}

The Open Observatory of Network Interference (OONI) project was started in 2012 as a non-profit open-source software project aimed at identifying and documenting internet censorship around the world \cite{ooniAbout}. The OONI organization openly publishes measurements and provides a public archive on network interference from across the world. 

\subsection{Data-Collection}

\subsubsection{Web Connectivity Test}

The Web Connectivity test determines if, and how, access to a specific website may be blocked. To do this, OONI Probe performs several checks from the network where the test is run and compares the results with measurements collected from a control network where censorship is not expected. If the measurements differ significantly, censorship techniques are likely used on the local network. This test is designed to perform the four different actions: Resolver Identification, DNS Lookup, TCP Connect, HTTP GET Request.

The Web Connectivity test begins by identifying the DNS resolver in use on the network. It achieves this by sending DNS queries to special domains, which disclose the resolver’s IP address. Once the resolver is identified, the test performs DNS lookups to determine which IP addresses (and potentially other host names) are mapped to the tested domain. After collecting that information, the test attempts to establish a TCP session on port 80 or port 443, depending on whether the URL uses HTTP or HTTPS. Finally, once the TCP connection is successful, the test sends an HTTP GET request to the server hosting the website; under normal circumstances, the server will respond with the requested webpage content \cite{ooniConnectivityTest}.

\subsubsection{Circumvention Test}

The circumvention test is used to check weather Psiphon, Tor, or RiseupVPN are blocked on a given network. These are tools used to circumvent censorship by utilizing VPN, SSH, and HTTP proxy technologies. 

The Psiphon VPN serves as a tunnel that enables you to circumvent censorship by connected you to an uncensored portion of the internet \cite{ooniPsiphonTest}. The Psiphon test first uses Psiphon’s own code to establish a Psiphon tunnel. After the tunnel is created, the test attempts to load a webpage to see if Psiphon actually works for accessing the internet. If the tunnel is successfully set up and the webpage loads, Psiphon is functioning on the tested network and can bypass censorship. If the tunnel is established but the webpage does not load, Psiphon is blocked in some way, preventing access to online resources. Finally, if the test cannot even create the Psiphon tunnel, it indicates that Psiphon is completely blocked on that network \cite{PsiphonTestGitHub}.

The Tor Test \cite{TorTestABOUTOONI} automatically checks whether Tor is accessible in a given network by examining the reachability of core components such as Tor directory authorities, OR ports, and obfs4 bridges. It first attempts to retrieve the Tor consensus from directory authorities, then tries to connect to OR ports (including those of directory authorities) via a TLS handshake, and finally tests obfs4 bridges through an obfuscated handshake. If all of these steps succeed, Tor is likely usable in the tested network (unless it is blocked in ways not covered by the test). If any step fails, Tor may be blocked and therefore unavailable on that network \cite{TorTestGitHub}.

The RiseUpVPN test evaluates if the bootstrap servers used during the self-configuration of the VPN clients can be reached. The test also checks if RiseupVPN’s gateways can be reached on different ports and transports \cite{RiseUpVPNTest}. This test was contributed by the LEAP collective \cite{leapLEAPEncryption}.

\subsubsection{Instant Messaging Test}

The Instant Messaging test is used to check weather WhatsApp, Facebook Messenger, Telegram, and Signal are blocked on a given network.

The Whatsapp test attempts to determine is there is any interference or blockage of it's App or Web Interface. To to this, the OONI probe attempts to perform an HTTP GET request TCP Connection, and DNS lookup to WhatsApp's enpoints. These include the endpoints used by the WhatsApp mobile app, the registration service, and the web interface \cite{ooniWhatsAppTest}. To conduct these tests, the OONI probe attempts to open TCP sockets towards WhatsApp endpoints on Ports 443 and 5222. If these connections fail or are rejected, it is seen as an indicator of blockage at the TCP level. The probe then verifies if the DNS resolution returned a valid IP address that is registered to WhatsApp. If the resolved IP address does not belong to WhatsApp, it can indicate DNS level blocking or tampering. And to check if the WhatsApp registration service is working correctly, an HTTP GET request is sent to the URL \url{https://v.whatsapp.net/v2/register}. The request is considered successful if there is no DNS, TCP connect, TLS (Transport Layer Security), or I/O error \cite{WhatsAppTestGitHub}. 

The Facebook Messenger Test is used to examine the reachability of the service within a tested network. The OONI probe begins by attempting to perform a TCP connect and DNS lookup to facebook's endpoints \cite{ooniFacebookMessenger}. The test verifies if Facebook Messenger endpoints resolve to consistently known IPs and if it's possible to establish TCP connections to them on port 443. For each endpoint tested, an A lookup for the domain name is performed and it is considered consistent if the IP is inside of a netblock linked to the \textit{Facebook Authonomous System Number} (AS32934) \cite{FacebookTestGitHub}.

The Telegram Test is used to examine the reachability of Telegram's app and web version within a tested network. The telegram access points (DCs) are those used by the desktop client, and they have six unique IP addresses. The test establishes a TCP connection to all of the access point IP addresses and attempts to send a POST HTTP request to each of them. If all TCP connections on ports 80 and 443 fail, Telegram is considered to be blocked at the TCP level. Otherwise, Telegram is considered to be working as intended \cite{TelegramTestGitHub}. 

The Signal Test is used to measure the reachability of the Signal messaging app within a tested network. The test checks if it is possible to establish a TLS connection and send an HTTP GET request to the Signal server endpoints \cite{ooniSignalTest}. A DNS query to \url{uptime.signal.org} is also performed to check if the backend servers are down \cite{SignalTestGitHub}. 

\subsubsection{Middlebox Test}

A Middlebox is a computer networking device that transforms, filters, and manipulates traffic for purposes other than packet forwarding. These include network address translators, load balancers, and deep packet inspection (DPI) devices. The presence of Middleboxes can lead to evidence of censorship and/or traffic manipulation, but it can also be indicative of a less malicious intent, such as network caching.

The OONI Middlebox test consists of two main operations: HTTP Header Field Manipulation and HTTP Invalid Request Line. The HTTP header field manipulation test emulates an HTTP request towards a server, but sends HTTP headers that have variations in capitalization. These requests are sent to a backend control server which send back any data it receives, and if these requests return exactly as we sent them, it is assumed there is no middlebox present. If the alterations of the headers come back normalized, it can be assumed that there was packet manipulation of some kind, leading to the confirmation of presence of Middleboxes It is worthy to note that false negatives can happen in this test, as some ISPs use highly sophisticated software that can disguise the presence of Middleboxes \cite{ooniHTTPHeader}.  

The HTTP Invalid request line test sends an invalid HTTP request to an echo service listening on the standard HTTP port, rather than a valid one. If the request is returned to the user exactly as it was sent, it can be concluded that there is no evidence of the presence of a Middlebox. However, it is possible that this invalid request can be intercepted by a Middlebox that triggers an error that is sent back to the probe. This is evidence that there is a Middlebox present in the network. It is worthy to note that false negatives are possible as some ISPs use highly sophisticated software that is designed not to trigger such errors \cite{ooniHTTPInvalid}. 

\subsection{Data Collection and Transparency}

All results from OONI Probe tests are automatically sent to OONI's servers and published on the OONI explorer. This transparency ensures that anyone can explore the measurements for themselves. OONI aggregates measurements by country, time, and type of test. It highlights "confirmed" cases of blocking when there is strong enough certainty in the test result, but it also publishes anomalies that might be considered false positives. 

The OONI team also work to release comparative analysis and real-time alerts for significant internet censorship related events. This would include events such as a sudden surge in social media blockage, or a complete drop off of internet traffic in certain areas. The OONI Mesaurement Aggregation Toolkit (MAT) can be used to visualize these events and potentially identify emerging trends. 

\section{Data Collection}

\subsubsection{Ireland}

To collect network data in Ireland, the OONI CLI was installed on a MacBook Air M2 located in Ireland. The OONI probe was installed based on the CLI instructions found on the OONI website \cite{OONISCLI}. The OONI probe by default does not run automatic tests on the Mac version of the CLI, and this was manually enabled during the installation process. All tests run in Ireland were using the providers HEAnet CLG (AS1213), and Liberty Global B.V. (AS6830).

\subsubsection{Iraq}

To collect network data in Iraq, a Virtual Machine was set up using the provider \textit{LightNode} \cite{lightnodeLightNodeGlobal} on a cloud server located in Baghdad. The cloud server was running Linux Ubuntu version ...PUT VERSION HERE...The OONI probe was installed using the CLI instructions available on the OONI website \cite{OONISCLI}. By default, the OONI probe runs censorship tests in the background once per day, and saves that data in a specific directory, as well as publishes the data publicly. All tests run in Iraq were using the provider Kaopu Cloud HK Limited (AS138915)

\subsubsection{Overlapping Information}

In addition to these automated tests, manual tests were conducted once per day to collect data. The OONI CLI user guide provided the specific commands to carry out these tests as well as how to run tests on specific files or test sets \cite{ooniUserGuideCLI}. The first and second day of data collection period used the comprehensive OONI test suite, which ran every test available, including 2200 websites. This broad test suite was used to identify blocked websites that could be added to a smaller test set of websites.

Following the initial testing, a set of 127 websites was collected. This test set included blocked websites from the broad OONI test set, some websites from the most known blocked websites worldwide \cite{blocksiteMostBlocked}, the top 50 most visited websites in Ireland \cite{top50irishwebsites}, and the top 50 most visited websites in Iraq \cite{top50IraqWebsites}. This test set was used in both countries for a span of 9 days.

Each website on the list was put into a category. The number of websites in each category can be found in the table below.

\begin{table}[H]
\centering
\caption{Number of Websites in Each Category}
\begin{tabular}{lc}
\toprule
\textbf{Category} & \textbf{Number of Websites} \\
\midrule
Uncategorized & 30 \\
Piracy / Streaming / File Sharing & 20 \\
News / Media & 20 \\
Adult Content & 20 \\
Creative / Educational / Misc & 15 \\
General / National Services & 9 \\
Streaming / Social Media & 9 \\
Religious & 4 \\
VoIP / Communication & 2 \\
Gambling & 2 \\
Email/Privacy Tools & 2 \\
Adult / Alcohol & 2 \\
LGBTQ+ & 1 \\
AI / Technology & 1 \\
\bottomrule
\end{tabular}
\label{tab:website_category_amount}
\end{table}

To parse the OONI data, a python script was written that took in the raw bash/cmd output and organized the results of the test into a neatly formatted CSV file. This made data analysis much easier to complete. Each of the four main tests were recorded as raw bash/cmd output and parsed into CSV files for analysis. All results and OONI links are available on this project's public GitHub Repository, which can be found in the appendix of this report.

\section{Challenges \& Limitations}

While the OONI probe and its tests give a good baseline of what internet censorship looks like in a country, it may be difficult to draw definitive conclusions based off of these tests alone. The list of websites used in this work, while tailored to fit these two countries, still has massive gaps, and may not show the entire picture of what content is blocked in each country. As of 2024 there are about 200 million active websites on the internet \cite{digitalsilkManyWebsites}. Testing every single one in both countries would provide a much more comprehensive view and potentially more concrete results, but this approach is not practical. Therefore, this work is limited in its actual data collection, but with the aid of past data and known censorship environments, conclusions can still be reached.


