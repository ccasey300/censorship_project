\chapter{Security and Privacy}

This section addresses security and privacy concerns involved with operating the OONI probe in relation to this work while considering both the technical aspects and the broader legal or regulatory aspects. The comparison of censorship between Ireland and Iraq is significant. While using the OONI probe within both countries comes with its own risk, the use of the probe in Iraq carries much more concern when it comes to security and privacy. The environment in Iraq is significantly more dangerous with ongoing government surveillance, frequent shutting down of social media sites, and the risk of authorities considering unauthorized data-gathering activities as suspicious. All these factors indicate the need to carefully plan where, how, and why measurements are taken, as well as how resulting data will be stored.

\section{OONI}

Although OONI strives to minimize the collection of personal data, its measurements are published openly, which may inadvertently disclose approximate locations and times when tests occurred \cite{ooniOONIPrivacyAndSecurity}. If the individual running OONI is tied to a VM in Iraq with an IP address, local authorities or ISPs might link test activity back to the source. This risk is particularly heightened if the probe is frequently connecting to or testing politically sensitive, banned, or controversial websites. In a high-censorship environment, repeated network tests can attract attention and might be interpreted as an intentional challenge to government policies.

\section{Iraq Virtual Machine}

When deploying a VM in Iraq, the potential security and privacy risk increases due to the possibility that authorities or other outside sources might attempt to compromise the server. The government of Iraq might be motivated to confiscate or check the contents of the VM in order to identify individuals who are actively monitoring sensitive network interference. It is also possible that the hosting provider itself can be forced to give logs, user connections, or site testing targets, which removes all privacy the user has. To avoid this, one should ensure that no personal data is used on this VM, and tools for circumvention are used to encrypt the origin of the user accessing the VM.

Even beyond direct government intervention, there is the risk of third-party hacks or malware injection. A public and well known measurement platform like OONI will attract hackers looking to disrupt users of this tool or introduce malware that will capture all incoming and outgoing traffic. In Iraq, the network infrastructure might already contain middleboxes or deep packet inspection systems that are actively filtering or manipulating data. These devices sometimes disrupt the traffic generated by the OONI probe measurements, leading to manipulated data to be collected.

\section{Legal Risks}



