\chapter{State of the Art}
\section{Censorship Mechanisms}

\subsection{IP and DNS Blocking}

\subsection{Deep Packet Inspection (DPI)}

\subsection{Content Manipulation}

\section{Ireland}

\subsection{Censorship in the Past}

According to a report from the United States Department of State in 2011, it was found that there were no government restrictions on access to the internet or that the government actively monitored email or internet chatrooms \cite{stateTechnicalDifficulties}.

The Irish government engages in censoring or blocking the distribution of pirated copryrighted material. In 2009, the Irish Telecom Company, EIRCOM, blocked its customers from accessing the website \textit{The Pirate Bay}. The Pirate Bay is a Swedish website which provides links to copyrighted material. The website was hit with a lawsuit from major record labels and many ISPs around the world agreed to block access to the website as part of the settlement. However, not all Irish ISPs complied. The cable TV operator UPC announced that it would not comply \cite{irishtimesEircomBlock}. 

In alignment with international agreements, the Irish Government blocks access to websites that contain illegal content, such as Child Sexual Abuse Material (CSAM). The government has setup a hotline that allows citizens to anonymously report websites that they suspect contain illegal content, called hotline.ie \cite{hotlineAboutx2013}.

In contrast to other EU countries, Ireland does not have a broad government-mandated filtering system. They instead have the power through the Irish courts to mandate Irish ISPs to block certain websites. In addition, Irish ISPs may voluntarily enforce content filtering and website blocking in alignment with Irish content law.

Up until 2014, Ireland and other EU countries followed data retention laws, which required ISPs to store metadata for law enforcement purposes. In 2014, the European Court of Justice struck down the directive, which led to a change in this law in Ireland \cite{DataRetentionInvalid2014}. After this change, Ireland enacted the \textit{Communications (Retention of Data)(Amendment) Act 2022} \cite{irishlegalDataRetention}. This legislation allows for the general and indiscriminate retention of communications traffic and location data on the grounds of national security, where approved by a judge.

\subsection{Current Censorship}

As a whole, Ireland's censorship efforts are limited and specific. The government and ISPs target mainly illegal and pirated content. Some specific websites that have been blocked include 1337x, Eztv, BMovies, GoMovies, Putlocker, Rarbg, WatchFree, and Yts \cite{siliconrepublicMovieIndustry}. However, piracy websites are still widely accessible in Ireland.

It seems that Ireland has also rolled back blocks on some websites, such as Russian News outlets. Previously, the domain russia.tv, was blocked in Ireland. But as of 2025, it is able to be partially accessed. Based on data from the OONI project, there is evidence of TCP/IP blocking of this domain in Ireland. Based on the findings from OONI, this domain is able to be accessed when EIRCOM's root DNS server (AS5466, IP: 86.47.80.38) is used, but is blocked when accessed through Cloudflare's DNS server (AS14593, IP: 172.69.193.80).

\centerline{\includegraphics[width=480pt]{Griff/Latex/TCD SCSS CAPSTONE/Literature Review/Eircom Access russiatb.jpg}}

\centerline{\textit{Figure 1.1, EIRCOM DNS test for Russia.tv on OONI probe}}

\centerline{\includegraphics[width=480pt]{Griff/Latex/TCD SCSS CAPSTONE/Literature Review/CloudFlare Block Russiatv.jpg}}

\centerline{\textit{Figure 1.2, CloudFlare DNS test for Russia.tv on OONI probe}}

\centerline{\includegraphics[width=480pt]{Griff/Latex/TCD SCSS CAPSTONE/Literature Review/RussiaTV search OONI.jpg}}

\centerline{\textit{Figure 1.3, Russia.tv domain search on OONI}}

\section{Iraq}

\subsection{Censorship in the Past}

Most people in Iraq did not have access to the internet until the mid 2010s. This section might be redundant as not much has changed. 

\subsection{Current Censorship}

In a 2023 report from the United States Department of State, it was found that the government of Iraq restricted or disrupted access to the internet and censored online content, in conjunction with monitoring private online communications without appropriate legal authority \cite{USDoSIraq2023}. The Iraqi government and the Kurdistan Regional Government (KRG) consistently engage in implementing internet outages during protests or times of unrest \cite{freedomhouseIraqFreedom}. In 2023, Iraqi officials implemented 66 internet outages, more than any other country in the world. Most, if not all, internet infrastructure is controlled and managed by the government. 

After the fall off Saddam Hussein's Regime in 2003, the internet became much more accessible and the information landscape was opened. However, the current-day Iraqi government occasionally blocks websites, and more often social media websites in order to maintain stability and control during times of unrest \cite{freedomhouseIraqFreedom}. During anti-government protests in 2019, the Iraqi government blocked access to Facebook, X (Formerly Twitter), WhatsApp, and Instagram. In protests in 2018, some users in Iraq found that they were unable to use VPNs to circumvent website blocking. The government routinely engages in the censoring and blocking of Pornography and Gambling websites on the guise of protecting their citizens from harmful content. 

\section{Censorship Circumvention Tools}

\subsection{The Tor Browser}

\subsubsection{The Tor Project Background}

The Tor Browser is built on a concept called \textit{Onion Routing}, which was developed in the 1990s by researchers at the United States Naval Research Laboratory. The goal of the project was to create a communication method where data is wrapped in multiple layers of encryption so that no point in the network could reveal the sender and receiver \cite{torprojectProjectPrivacy}. Originally, the United States Government used the Tor network to access potentially illegal websites anonymously, and transmit data. But because only the US Government was using it at the time, it was easy to tell who the single anonymous user was, when viewing the site logs. It would also have made Tor a target for bad actors, as they could be sure that all data being sent over the network was related to the United States Government/Military.

To stop this from happening, the US Government released Tor to the public in the early 2000s, and later it became the Tor Project, a non-profit organization funded by the United States that develops and maintains the Tor software. 

\subsubsection{Technical \& Circumvention Information}

Internet traffic sent over the Tor network is encapsulated in multiple layers of encryption. Think of your data as a letter that is placed inside several envelopes. Each node in the network removes one envelope, revealing only the information necessary to pass the message along to the next node. To do this, the Tor browsers sends your data through at least three nodes, and the pathway of these nodes are randomly constructed and reconstructed during your session \cite{dingledine2004tor}.

\centerline{\includegraphics[width=480pt]{Griff/Latex/TCD SCSS CAPSTONE/Literature Review/How tor works.jpg}}

\centerline{\textit{Figure 1.4, How the Tor Network Works}}

Tor is a great tool to combat censorship. Tor's distributed architecture of nodes makes it resilient against localized censorship efforts. In countries where the Tor network is blocked, users are able to use "Bridges", which are Tor nodes that are not listed publicly. Using a bridge address allows for the user to connect to the network covertly \cite{torprojectBRIDGESProject}. Users can also avail of "Pluggable transports", which transforms Tor traffic to look like regular network traffic. This method can help circumvent censorship in regions that use \textit{Deep Packet Inspection} (DPI) and other forms of advanced internet censorship \cite{torprojectCIRCUMVENTIONProject}.

\subsection{VPNs}
