\chapter{Introduction}

\section{Internet Censorship and Privacy}

The primary aim of this work is to identify and compare internet censorship methods between Ireland and Iraq. 

\subsection{Privacy}

User Privacy across the internet is directly tied to censorship efforts from different regimes. Censorship often involves the state or corporate monitoring of internet users, and governments that impose censorship frequently justify it using security concerns while often violating privacy rights in the process. In countries where censorship is highly enforced, using anonymity tools to circumvent censorship can protect the right to free expression and access to information. For instance, the \textit{Human Rights Watch} advises people in China to make use of the Tor Browser to avoid abuses by the state \cite{Privacy2017}. 

Based on a meta-analysis of studies related to internet privacy concerns, privacy literacy, and the adoption of privacy-protective measures, it was found that there is no strong correlation between national privacy laws and protective behaviors \cite{baruh2017online}. This suggests that individuals do not rely on legal protections in their country, and more often take privacy into their own hands. It was also found that culture did not impact the use of privacy-protective behaviors in different countries. 

While it may be easy to think censorship is only prevalent in non-western countries, such as China or Russia, it can also happen in democratic states. Weak privacy protections can lead to surveillance capitalism, where companies act as de facto censors by shaping information flows based on user data \cite{schwartz1999internet}. For example, during the COVID-19 pandemic in the United States, it was recently revealed that Meta (formerly Facebook) was asked to censor certain information regarding COVID-19 \cite{pbsZuckerbergSays}. The United States Government and Meta actively engaged in the censorship of the people's right to free speech and expression, as humor and satire was also removed from the platform.

\subsection{Background}

This section will talk about internet censorship across the world and give a brief intro into the differences by general region

\subsection{Global Censorship}

\section{Project Goals}

The aim of this project is to ...


