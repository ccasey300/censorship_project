\chapter{Literature Review}

\section{Introduction}

Write an introduction outlining the reasons for lit review... Briefly elude to some high level differences between Ireland and Israel.

67.4%

\section{Literature Review Methodology}





\section{Findings of the Literature Review}
\subsection{Ireland Historically (Griff)}
According to a report from the United States Department of State in 2011, it was
found that there were no government restrictions on access to the internet or that the
government actively monitored email or internet chatrooms (6).
The Irish government engages in censoring or blocking the distribution of pirated
copryrighted material. In 2009, the Irish Telecom Company, EIRCOM, blocked its
customers from accessing the website The Pirate Bay. The Pirate Bay is a Swedish
website which provides links to copyrighted material. The website was hit with a
lawsuit from major record labels and many ISPs around the world agreed to block
access to the website as part of the settlement. However, not all Irish ISPs complied.
The cable TV operator UPC announced that it would not comply (7).
In alignment with international agreements, the Irish Government blocks access to
websites that contain illegal content, such as Child Sexual Abuse Material (CSAM).
The government has setup a hotline that allows citizens to anonymously report
websites that they suspect contain illegal content, called hotline.ie (8).
In contrast to other EU countries, Ireland does not have a broad
government-mandated filtering system. They instead have the power through the
Irish courts to mandate Irish ISPs to block certain websites. In addition, Irish ISPs
may voluntarily enforce content filtering and website blocking in alignment with
Irish content law.
Up until 2014, Ireland and other EU countries followed data retention laws, which
required ISPs to store metadata for law enforcement purposes. In 2014, the European
Court of Justice struck down the directive, which led to a change in this law in
Ireland (9). After this change, Ireland enacted the Communications (Retention of
Data)(Amendment) Act 2022 (10). This legislation allows for the general and
indiscriminate retention of communications traffic and location data on the grounds
of national security, where approved by a judge.

\subsection{Ireland Today (Griff)}
As a whole, Ireland’s censorship efforts are limited and specific. The government and
ISPs target mainly illegal and pirated content. Some specific websites that have been
blocked include 1337x, Eztv, BMovies, GoMovies, Putlocker, Rarbg, WatchFree, and
Yts (11). However, piracy websites are still widely accessible in Ireland.
It seems that Ireland has also rolled back blocks on some websites, such as Russian
News outlets. Previously, the domain russia.tv, was blocked in Ireland. But as of
2025, it is able to be partially accessed. Based on data from the OONI project, there is
evidence of TCP/IP blocking of this domain in Ireland. Based on the findings from
OONI, this domain is able to be accessed when EIRCOM’s root DNS server (AS5466,
IP: 86.47.80.38) is used, but is blocked when accessed through Cloudflare’s DNS server (AS14593, IP: 172.69.193.80).






\subsection{Israel Historically}


\subsection{Israel Today}




\section{Analysis: Ireland vs Israel}


\subsection{Similarities}



\subsection{Differences}



\section{Conclusions}

