\chapter{Introduction}

\section{Research Motivation}

Since its inception, the Internet has served a vast user base that wishes to communicate with one another and spread information. By design, the Internet is a platform that should provide unfiltered content to users. In many cases this content may otherwise be inaccessible by traditional media outlets such as radio or TV. Originally designed to aid government researchers share information, its open and transparent foundation has since changed. Across the globe, governments and other entities are censoring the internet by network manipulation, legislative pressure or otherwise. This is a global threat to fundamental internet user rights and should be treated as such. It is for these reasons that this area ought to be investigated more thoroughly.

Although censorship of certain content (CSAM, pirated entertainment) is widely considered appropriate, normalising this has far reaching consequences on user privacy and free speech. The importance of establishing a quantitative approach to measuring internet censorship cannot be overstated as users are unaware of invisible content in most cases. As a result, the state has a large influence over what ideas can propagate within its borders. Various open-source and community led projects aimed at addressing this issue. Notable examples include the Tor project, OONI, Tails OS and others. However, it is coming to light that this technology is becoming deprecated. In 2022, German police were able to make an arrest after de anonymising Tor traffic using timing analysis. [1]. This highlights the large dichotomy between what users believe governments are capable of and reality. 



For the above reasons, the increasingly pervasive censorship done by governments and corporations around the world is concerning.  


\subsection{Practical Implications}
\subsection{Awareness \& Transparency}
\subsection{Problem Statement}

\section{Background}
\subsection{Global Internet Censorship}
Experts suggest that censorship on the internet is increasing at an alarming rate. “The majority 
of countries that censor content do so across all four themes, although the depth of the filtering 
varies. The study confirms that 40 percent of these 2,046 websites can only be reached by an 
encrypted connection (denoted by the "HTTPS" prefix on a web page, a voluntary upgrade from 
"HTTP").” [4] It is also clear that more and more countries are viewing this as a necessary 
solution to the unique problems they have. Whether this is appropriate or not, it is happening, 
and users should be aware of this. 

Governments have a vested interest in maintaining control over telecommunications industries 
and public internet use. Whether protecting state secrets, preventing cyber crime piracy or acts 
of terrorism, insulating from perceived negative influence, aiding in the creation of propaganda 
or otherwise; a large majority of governments choose to exercise inordinate control over the 
information available to its public.  

As more governments and entities began to engage in this, it became increasingly important to 
hold them accountable. As a result, the ‘Enemies of the Internet’ list was devised. It contains 
the governments and entities that actively engage in the repression of online freedoms,  in the form of censorship and surveillance. As of 2014, there were 19 governments that fit this criterion but by now this number has likely increased. [5] Traditionally, censorship involved monitoring a handful of media and cutting undesirable content, potentially replacing this with a message more in line with the agenda and norms of the locale. However, with the advent of the internet, this distribution of information became decentralised and thus allowed for more expression and freedom in the content consumed by a user. As a result, censorship has become more difficult to conduct, but potentially easier to get away with. Nowadays, governments leverage points of control, network-level filtering and many other techniques to block undesirable content.

\subsection{User Privacy}
\subsection{Censorship vs. Surveillance}
\subsection{Legislation}
Governments can enforce censorship directly through ISPs, tech companies and social media 
platforms by creating new legislation or simply mandating content be removed. This is used to 
deplatform individuals and movements during periods of unrest. This is also done in app stores, 
shutting down entire platforms that are deemed problematic. 



\section{Project Scope}
\subsection{Project Objectives}
Below is an outline of the objectives completed during the duration of the project:

-   To conduct a literature review to identify and evaluate existing censorship measurement 
tools with a focus on OONI. 

-   To understand how and why censorship is conducted in these countries and how it can be 
measured. 

-   To collect data using OONI and other sources for both countries. Use historical datasets 
as well as rerunning tools for up to date data. 

-   To conduct a comparative analysis between the two countries’ datasets. 

-   To consider ethical implications of the research early so as to ensure compliance. 

-   To set up VMs in Israel in order to establish ground truth. 

-   To present high-level findings about the two countries approach to censoring the 
experience of their internet users, make conclusions about the attitudes and values 
present in each locale based on the data collected. 

-   To understand more about the unique situations of both Ireland and Israel, and how 
censorship is used by the state considering this.

\subsection{Core Research Questions}
\subsection{Data Collection \& Analysis Tools}

\section{Ethical Considerations}
\subsection{GDPR and Data Privacy}
\subsection{Potential Risks}