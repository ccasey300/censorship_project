\chapter{Introduction}

\section{Research Motivation}

Since its inception, the Internet has served a vast user base that wishes to communicate with one another and spread information. By design, the Internet is a platform that should provide unfiltered content to users. In many cases this content may otherwise be inaccessible by traditional media outlets such as radio or TV. Originally designed to aid government researchers share information, its open and transparent foundation has since changed. Across the globe, governments and other entities are censoring the internet by network manipulation, legislative pressure or otherwise. This is a global threat to fundamental internet user rights and should be treated as such. It is for these reasons that this area ought to be investigated more thoroughly.

Although censorship of certain content (CSAM, pirated entertainment) is widely considered appropriate, normalising this has far reaching consequences on user privacy and free speech. The importance of establishing a quantitative approach to measuring internet censorship cannot be overstated as users are unaware of invisible content in most cases. As a result, the state has a large influence over what ideas can propagate within its borders. Various open-source and community-led projects aimed at addressing this issue. Notable examples include the Tor project, OONI, Tails OS and others. However, it is coming to light that this technology is becoming deprecated. In 2022, German police were able to make an arrest after de-anonymising Tor traffic using timing analysis \cite{TorDeanonymization}. This highlights the large dichotomy between what users believe governments are capable of and reality. 

INSERT ABOUT USER PRIVACY

For the above reasons, the increasingly pervasive censorship done by governments and corporations around the world is concerning. The sheer number of individuals affected by internet censorship and the lack of transparency are strong motives for more extensive research.

\section{Background}
\subsection{Global Internet Censorship}
Experts suggest that censorship on the internet is increasing at an alarming rate. “The majority of countries that censor content do so across all four themes, although the depth of the filtering varies. The study confirms that 40 percent of these 2,046 websites can only be reached by an 
encrypted connection (denoted by the "HTTPS" prefix on a web page, a voluntary upgrade from "HTTP").” \cite{Zittrain2017Censorship} It is also clear that more and more countries are viewing this as a necessary solution to the unique problems they have. Whether this is appropriate or not, it is happening, and users should be aware of this. 

Governments have a vested interest in maintaining control over telecommunications industries and public internet use. Whether protecting state secrets, preventing cyber crime piracy or acts of terrorism, insulating from perceived negative influence, aiding in the creation of propaganda or otherwise; a large majority of governments choose to exercise inordinate control over the 
information available to its public.  

As more governments and entities began to engage in this, it became increasingly important to hold them accountable. As a result, the ‘Enemies of the Internet’ list was devised. It contains the governments and entities that actively engage in the repression of online freedoms, in the form of censorship and surveillance. As of 2014, there were 19 governments that fit this criterion but by now this number has likely increased. \cite{RSFEnemiesInternet2014} Traditionally, censorship involved monitoring a handful of media and cutting undesirable content, potentially replacing this with a message more in line with the agenda and norms of the locale. However, with the advent of the internet, this distribution of information became decentralised and thus allowed for more expression and freedom in the content consumed by a user. As a result, censorship has become more difficult to conduct, but potentially easier to get away with. Nowadays, governments leverage points of control, network-level filtering and many other techniques to block undesirable content.

\section{Project Scope}
Below is an outline of the objectives completed during the duration of the project:

-   To conduct a literature review to identify and evaluate existing censorship measurement tools with a focus on OONI. 

-   To understand how and why censorship is conducted in these countries and how it can be measured. 

-   To collect data using OONI and other sources for both countries. Use historical datasets as well as rerunning tools for up to date data. 

-   To conduct a comparative analysis between the two countries’ datasets. 

-   To consider ethical implications of the research early so as to ensure compliance. 

-   To set up VMs in Israel in order to establish ground truth. 

-   To present high-level findings about the two countries approach to censoring the 
experience of their internet users, make conclusions about the attitudes and values 
present in each locale based on the data collected. 

-   To understand more about the unique situations of both Ireland and Israel, and how 
censorship is used by the state considering this.

\section{Privacy \& Security Concerns}
The following section contains information regarding the privacy and security concerns associated with the completion of the dissertation. This was completed in conjuncture with an assignment given in the CSU44302 Security and Privacy module. 
In writing a dissertation, it is crucial to consider the potential impacts of the research. This document discusses the security and privacy concerns associated with researching internet censorship. Initially, theoretical vulnerabilities will be explored. Specific cases such as Israel and Ireland will then be analyzed. Finally, a practical perspective will examine realistic security and privacy concerns, along with relevant case studies.
The Open Observatory for Network Interference (OONI) will be used to gather data on internet censorship. OONI provides a comprehensive set of tests that can be run globally, offering insights into censorship situations. Since 2012, OONI has gathered over 2.5 billion measurements across 242 countries.

A virtual machine will be used to establish ground truth in Israel, running OONI CLI locally. The provider chosen, \textit{interhost.co.il}, has a strong track record regarding data integrity and security; however, further scrutiny is necessary. Below is an audit of the software used during research.

\subsection{OONI \& OONI Probe}
OONI, the Open Observatory for Network Interference, is a non-profit free software project that aims to empower decentralized efforts in documenting internet censorship worldwide \cite{OONIAbout}. Released under the TOR project in 2012, it has maintained a strong reputation for data integrity and reliability.

Their positive track record is emphasized by the claim: \textit{“To our knowledge, no OONI Probe user has ever faced consequences as a result of using our software.”} \cite{OONIRisks}. The success of OONI is critically dependent on users conducting tests without repercussions. However, OONI outlines several scenarios in which running their probe may be unwise. This includes users residing in countries with a history of prosecuting similar activities, surveillance concerns, or legal restrictions on accessing content. Users who fall into one or more of these categories should be wary of the potential risks. In this context, operating in Ireland with no reason to believe I am under surveillance, I am considered a low-risk user.

Released in 2017, the OONI Probe is a mobile app and software designed to test internet censorship. Users can install and run this software, contributing to the growing dataset in the OONI database. OONI's mission is to \textit{“ensure a free and open internet by increasing transparency of internet censorship worldwide.”} While no user appears to have faced repercussions from using the OONI probe, this may lead to a false sense of security or incorrect assumptions about anonymity. As emphasized in OONI’s onboarding quiz, probe tests can be visible on the network.

\subsection{SSH \& Virtual Machine (Interhost.co.il)}
A virtual machine (VM) emulates a computer system. It is a file (.img) that contains instructions to create a virtual environment, leveraging physical PC resources. The provider was chosen based on location availability, with Interhost offering a machine in Tel Aviv. The shared nature of resources introduces vulnerabilities. The number of users sharing the same hardware is unknown, so file-sharing precautions were taken. Storing sensitive information on this VM may be unwise for these reasons.

SSH is a cryptographic protocol that allows users to securely and remotely control a machine over an unsecured network. It employs a client-server model with public-private key pairs for encryption and password authentication. In this project, SSH was used to remotely control the VM in Tel Aviv. Upon generating key pairs, authentication was established, and the VM became accessible. Provided secure settings are maintained and the private key is never shared, SSH is a reliable and trustworthy protocol \cite{SSHManual}.

\subsection{Other Security and Privacy Considerations}
Personal safety risks in researching internet censorship must be addressed. My supervisor highlighted that selecting a comparison country required more than just finding a contrast. Publishing documents that critique government censorship has historically been risky \cite{JulianAssange, EdwardSnowden}. However, this research is relatively low-risk. Particularly compared to cases like Assange or Snowden. MIFTAH, an organization advocating open dialogue on the Israel-Palestine conflict, reported 310 press freedom violations from 2000-2003 \cite{MIFTAHReport2003}. Although Israel has a history of reprisals, these incidents were tied to conflict zones such as Gaza. This research is not of a whistleblowing nature, reducing potential risks.

Researching Israeli state-sponsored internet censorship inevitably intersects with ongoing conflicts. The thesis remains unbiased and non-political. Historical events are included only to provide accurate context for internet censorship analysis. While Israel’s military has strong ties to information control \cite{IsraelCensorship}, it is crucial that internet censorship is examined from an empirical lense.

The security and privacy considerations for this project required assessing all tools used. OONI’s privacy and security protocols appear robust. Its strong track record reinforces this confidence. SSH, as a long-established protocol, remains secure when best practices are followed. Potential consequences of researching this area are more extensive than initially anticipated. However, given my threat model, adverse effects are unlikely. Best practices will continue to be followed to ensure nonpartisan, data-driven research.


