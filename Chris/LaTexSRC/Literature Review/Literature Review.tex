\chapter{Literature Review}

\section{Introduction}

The purpose of this literature review is to survey and consider published work regarding internet censorship globally, with a particular focus on that of Ireland and Israel. Legislation, important events and other notable areas will be discussed and compared in order to gain a greater understanding the differences between internet censorship experienced in Israel and Ireland. Internet censorship is constantly evolving as it competes with privacy based tools in an 'Arms race' of sorts. This makes researching and understanding how censors achieve their purpose of particular importance. To properly understand the current situations faced by both Israeli and Irish citizens using the internet, a broad analysis of existing literature and ongoing research had to be considered. This section lays the groundwork for the thesis, detailing how both countries approach to internet censorship has evolved over the years. 

Internet usage is rising year on year globally as more users are free to surf the web. Our World in Data, an independent organisation that tracks internet usage statistics suggests that as of 2023, 67.4\% of the world was connected to the internet. \cite{owid-internet} This is a staggering number of individuals that is only set to increase. With more people relying on the internet for their livelihood, communication or otherwise, internet censorship is becoming a more pressing matter. It has also been noted previously that, based on OONI data, censorship is rising globally. This growth highlights the need for transparency and regulation surrounding user rights and privacy. 

As previously mentioned, a common misconception about the internet is that it's content is not manipulated. Another misconception held by many is that internet censorship occurs in few countries. This is also false, with censors increasing their restrictions continuously. According to Bischoff in his online article mapping internet censorship and geographies "This year we saw nearly 60 countries increase their internet censorship in some way, compared to 50 from last year’s study."\cite{bischoff2025internet} This is a troubling reality as internet content is increasingly being censored not just in authoritarian countries but by democratic states. 

\section{Literature Review Methodology}

In conducting this literature review, sources from a variety of mediums were used. Research was conducted primarily using the internet, focusing on academic papers and peer reviewed literature. Other sources like Trinity College Library, online articles and journals were also considered and sources were cross checked with relevant authorities. Information regarding country-specific legislation was taken from the official state-run website of those countries. Sources deemed potentially unreliable were placed under a higher level of scrutiny. It is worth mentioning that researching this area can be difficult as state level censorship is typically clandestine and covert. This sections aims at laying the foundation for discussing both countries' approach and sentiments towards internet technologies. 


\section{Findings of the Literature Review}
\subsection{Ireland Historically }
In researching the early days of internet adoption in Ireland, a fascinating story emerges. As will be discussed, academia paved the way for the internet infrastructure Irish users enjoy today. To research the early days of internet technologies in Ireland, a variety of sources were used. One that was particularly helpful but admittedly casual resource was internethistory.ie. \cite{InternetHistoryIE} Niall Richard Murphy documents a detailed account of Irish Internet history here. In order to verify this account and investigate specific cases, techarchives \cite{TechArchivesIE} was used. Other media and press sources such as RTE, The Irish Times and BBC were used in a similar fashion.

The internet in Ireland has come a long way since The Irish Sugar Company took delivery of the first computer in the country. "The Sugar Company paid £33,000 for the system." \cite{irelands_first_computers} Conservatism hampered development early on, however, as the cost of hardware decreased in the 60s Irish universities began investing in computers. Tech Archives suggests of early adopters, "their financial backing was minimal. They relied on students and volunteers to keep running. Their day-to-day operations were largely improvised and sometimes anarchic." \cite{internet_ireland_1987_97} This progress continued into the 80s and 90s as the potential became apparent. This progress was also driven by IT professionals who adopted Unix. EUnet was a collaborative effort that succeeded in becoming the "first public wide area network in Europe." \cite{internet_ireland_1987_97} The gateway was later relocated and managed by Trinity College Dublin. This advancement set the stage for the foundation of Ireland's first ISP by Cormac Callanan and Michael Nowlan, IEUnet. In a low-key email, Nowlan announced the arrival of the internet in Ireland, aptly stating "No gaurantees of reliable service are offered at present, it is quite likely that the line with go down at no notice. " \cite{tcd_scss_t_20160323_001} 

Though the Irish government played a passive role in this development, they deserve credit for the legislation they passed in 2000. The E-Commerce Act legitimised electronic transactions by asserting "electronic communications are deemed equivalent to written or oral communications." \cite{electronic_documents_legalguide} This seems trivial today, but it was a huge step in the right direction. Though Ireland was in many ways following EU trends, a strong emphasis was placed on privacy. The Irish Times released an article discussing how the legislation "appears to protect the privacy of communications far more than the laws of our neighbours." \cite{irishtimes_privacy_legislation} This is contextually impressive considering other institutions, such as the FBI, harked "the widespread use of robust non-recovery encryption will ultimately devastate our ability to fight crime and terrorism." \cite{irishtimes_privacy_legislation} 

Fast forward to 2002 and Eir has began rolling out broadband internet access to consumers. In 2005, Ireland surpassed 2 million internet users, denoting the successful adoption of the technology. At this point, the internet and its technologies had penetrated industry, academia and consumerism. This time marked a turning point in communications both in Ireland and globally.

\subsection{EU Compliance}
Aside from Irish legislation, there are EU directives that run in conjuncture such as the Digital Services Act, GDPR and more. These will be discussed in detail in this section. Let us first establish the primacy of EU law as echoed in the Irish Constitution. "As well as being superior to national law, some EU law has direct effect on its citizens." \cite{citizensinformation2025} The European Commission proposes laws that are sent to the European Commission and then the Council of the European Union to be approved by a qualified majority and passed, or rejected. \cite{europa2025} This procedure has lead to the passing of legislation that affects internet usage in Ireland. 

The first law to be discussed is General Data Protection Regulation (GDPR). This legislation is designed to protect user privacy and bolster data integrity. It highlights the acceptable procedures for handling user data and is used to police organisations. Penalties come in the form of fines, up to 20 million euros. \cite{gdprinfo2025} 
The second law that is worth mentioning is the Digital Services Act, which came into EU law in November of 2022, and was followed by the Irish law of the same name in 2024. \cite{enterprisegovie2025} \cite{irishstatutebook2024} This act addresses illegal content, disinformation and transparent advertising and thus, is of particular relevance to the research being conducted. 
The final example of EU legislation that has undoubtedly shaped internet censorship seen in Ireland is the Copyright Directive (2019). This legislation solidifies intellectual copyright law within the EU, providing some edge cases of where free use applies. \cite{EUCopyright} These three important pieces of legislation guide Ireland's internet censorship. 

It has been mentioned that compliance is a strong motivation for the Irish government to censor and due to EU law primacy. Now, let us focus on some notable real world examples of legislation being used to censor content online or otherwise reprimand nonconforming organisations. Four high profile cases of non compliance will be discussed as well as the outcomes in each case. 

\textbf{1. Google and the "Right to be Forgotten" (2014)}
In 2014, the Court of Justice of the European Union (CJEU) ruled in Google Spain SL v. Agencia Española de Protección de Datos. This salient case granted individuals the right to request the removal of outdated personal data from search engine results. \cite{google2014} This landmark ruling was based on the EU's data protection laws. The ruling had a profound impact on how search engines and online platforms handle personal data. Google, as the largest search engine, were forced to make stark changes in their browser's operation, reshaping online content management. The decision triggered similar discussions on privacy and free speech, creating a global precedent for data removal requests.

\textbf{2. YouTube and the EU Copyright Directive (2019)}
Under the EU Copyright Directive, Article 17 requires platforms like YouTube to prevent the upload of copyrighted content without permission. \cite{eu2019} This change forced YouTube to implement automatic content filtering systems. This mandated more stringent controls over user-uploaded videos, significantly impacting how online platforms handle user-generated content. Though it aimed to protect copyright holders, it also raised concerns about excessive censorship. Automatic filters could lead to the removal of legitimate content, raising a significant challenge in balancing copyright protection and free speech.

\textbf{3. Facebook and the GDPR Fines (2021-2024)}
In 2021, Meta (formerly Facebook) was fined €265 million by Ireland's Data Protection Commission (DPC) due to a data breach that exposed personal information of millions of users. \cite{facebook2023} The breach was linked to the company's failure to adequately protect user data, violating GDPR standards. Meta has faced a litany of fines at the hands of the EU and other regulating bodies. Meta has received fines of over 2 billion euros in 2024 alone. \cite{FBFines} This exemplifies the European Union's stringent enforcement of GDPR which holds companies accountable for safeguarding personal data. 



\textbf{4. Twitter's Non-Compliance with the Digital Services Act (DSA)}
In 2023-2024, Twitter (now X) faced scrutiny under the EU Digital Services Act (DSA) for failing to implement required measures to combat illegal content and misinformation. \cite{twitter2023} The platform was given deadlines to comply, including implementing more robust content moderation policies. This case highlighted the increasing regulatory pressure on tech firms to ensure their platforms are safe, free of illegal content, and accountable for user actions. Previously, X withdrew from a voluntary agreement to combat disinformation online. Despite this protest in the face of the Digital Services Act, legislators were quick to point out that X will still have to comply with EU standards. \cite{bhr2023}



\subsection{Ireland Today}
As of 2024, the Irish digital sector exceeded a valuation of \$50 billion, \cite{tradegov2025} Bolstered by strong international relations and low corporate tax, Ireland is an attractive location for any tech company looking to infiltrate the EU market. A corporate tax of 12.5\% \cite{revenueie2025} and a skilled and educated work force has proved tantalizing for tech giants. Ireland balances citizens privacy and safety with keeping tech giants satisfied. Over the years, this approach has proved very effective with the likes of Google, Facebook (Meta), Apple and Microsoft setting up EU bases in Ireland. It is clear that Ireland's tech industry is a booming financial success but it has faced criticism for its tax system. Criticism recently came at the hands of financier Howard Lutnick who described Ireland as his favourite "tax scam." \cite{archyde2025} He highlights concerns around the number of US pharmacological and tech companies operating in Ireland and the intellectual property acquired by Ireland as a result. Conjecture aside, it is clear that there are strong forces at play that influence the regulation of tech companies on Irish soil. This influence undoubtedly affects the internet censorship situation faced by Irish users and will be investigated in the context of case studies. Before this, let us briefly go over the legislation that affects this area. 

\textbf{The Defamation Act (2009)}
An early piece of legislation passed by the Irish government that addresses defamation both online and offline. This act describes defamation as "the publication, by any means, of a defamatory statement concerning a person to one or more than one person." \cite{defamation2009} Simply, this document aims to uphold an individuals right to a good name.

\textbf{Communication Act (2011)}
This act strengthened the regulation of electronic communication and focused primarily on issues like data retention, consumer rights and cybersecurity. This act articulates the legal framework for the monitoring and interception of certain online communications in the interest of law enforcement and national security. It mandates timelines for the safe retention and eventual destruction of sensitive consumer data. It also enables a senior Gardai to "request a service provider to disclose to that member data retained by the service provider." \cite{communications2011}

\textbf{Online Safety and Media Regulation Act (2022)}
This law targets harmful digital content like hate speech and misinformation. This empowers a newly appointed Coimisiun na Mean to regulate content that fits these criterion, addressing issues such as cyber-bullying and extremism. At the head of this organisation is the Online Safety Commissioner who is "responsible for the implementation of a binding online safety code." \cite{onlinesafety2022}

Now that we have some background on the legislation passed in Ireland relating to internet censorship, we can examine notable cases. 

\textbf{The Pirate Bay \& More (2009, 2013)}
In 2009, major media labels began pressuring six ISPs to block access to the Pirate Bay website. This came to a head in 2013, when the ISPs were ordered to block access by the High Court. \cite{piratebay_block2013} This was a momentous event as it marked the first use of updated EU copyright legislation. The website was and still is one of the most popular torrent sites globally, with total monthly visits exceeding 20 million today, according to Similar Web. \cite{piratebay_stats2025} Since the 2010s this trend of blocking piracy by judiciary process continued, with site like Eztv, Putlocker and GoMovies being blocked to this day.

\textbf{Ban on Russian State Media (2022)}
Following the Russian invasion of Ukraine in February of 2022, the EU released a directive \cite{eu_sanctions_russia_media} to block access to Russian state media websites. European Commission President Ursula von der Leyen announced "The state-owned Russia Today and Sputnik, and their subsidiaries, will no longer be able to spread their lies to justify Putin’s war." \cite{rt_ban2022} Following this, the EU imposed heavy sanctions on those who were found to be hosting versions of these websites. In response, Russia blocked a number of EU based media outlets, particularly those critical and vocal about the invasion. This included the Irish state-owned RTE and the Irish Times. \cite{euronews_russia_media_block2024}

\textbf{Social Media Under EU Law}
Under the aforementioned DSA and GDPR legislation, Ireland's Data Protection Committee has required platforms such as TikTok and X to remove non-compliant content. Platforms such as Meta, TikTok and X have faced legal trouble regarding misinformation and hate speech. A salient example is TikTok's refusal to submit a risk report prior to launching TikTok Lite in France, Spain and elsewhere. "Under the DSA, designated Very Large Online Platforms are obliged to submit a risk assessment report." \cite{ec_press_release_2024} 


\subsection{Israel Historically}
Since the early 2000s, internet access has become increasingly available in Israel. In a paper discussing internet usage in Israel, Fisher speaks of "an increase of 152\% in the number of Israeli households connected to the Internet during the period 2000–2005." \cite{FISHER2006984} 

An important aspect of this literature review was understanding how the Israel - Palestine conflict has shaped censorship of the press and the internet over the last few decades. To better appreciate the impact of this war, the Council of Foreign Relations (CFR) provides a brief overview of notable events. \cite{CFR2024Timeline} MIFTAH, an organisation promoting open dialogue on the Israel - Palestine conflict, released a summary of freedom of press violations for the years 2000-2003. They tabulated 310 separate incidents of press freedom violations during this time, with reporters and journalists consistently being victimised. \cite{Miftah2003Censorship} It is clear from these documents that it is dangerous to report on this conflict. It is also clear that wars such as this one inevitably affect the information available to users online.

In an archived document produced by the IDF in 2016, the details for mandatory conscription of Israeli citizens is described. \cite{MOIA2016IDF} This military draft has been ongoing since 1948 when Israel declared its independence. Men are required to serve 32 months while women serve 24. This policy, in combination with Israel's renowned intelligence operation has produced highly qualified cybersecurity professionals. 

Regarding Israel's freedom of press in the 2000s, the Internet Monitor, a data analysis and collection tool states: "Modern censorship of [press] operates through voluntary agreements between the military and the Israeli Committee of Daily Newspaper Editors. Even though these agreements lack full consent from media in the country, all media organizations operating in Israel must abide by the censor's decisions." \cite{internet_monitor_israel} Though this pertains to press rather than the internet, it shows a tendency by the state to block political content. This trend would go on to continue in the 2010s and 2020s. The Colomubia Journalism Review wrote an article in 2025 discussing the potential bias of Channel 14, a prominent right-wing media outlet in Israel. Channel 14 lends itself to nationalist and patriotic rhetoric and has been subject to criticism as a result. "Netanyahu’s relationship with Channel 14 goes back years, to the time when it was Channel 20, called the Heritage Channel." \cite{CJR2024IsraelChannel14} This serves as evidence to suggest the Israeli government has a strong grasp over its media. 

\subsection{Israel Today}
Today, a large majority of Israeli citizens have access to the internet. DataReportal, a website responsible for collecting and publishing global digital reports states "there were 8.51 million internet users in Israel at the start of 2024, when internet penetration stood at 92.1 percent." \cite{Digital2024Israel} It is also pertinent to mention Israel's booming cybersecurity industry. According to YL Ventures' recent report "In 2024, the Israeli cybersecurity industry demonstrated exceptional growth," receiving \$4B in funding, double that of 2023. The roots of this industry come as a direct product of the nation's fixation with intelligence and national security. 


Reporters Without Borders (RSF), responsible for the World Press Freedom Index, provide detailed reports pertaining to media censorship globally. They have ranked Israel as 101st in the world as of 2024 in this regard. This ranking is based on the level of freedom enjoyed by journalists and media. "Press freedom is defined as the ability of journalists as individuals and collectives to select, produce, and disseminate news in the public interest independent of political, economic, legal, and social interference and in the absence of threats to their physical and mental safety." \cite{rsf_israel} 

In a 2024 paper discussing digital diplomacy in the Israel - Gaza war, Othman asserts "Governments and non-state actors leveraged social media to influence international public opinion, while misinformation campaigns complicated the narrative, undermining trust in diplomatic channels." \cite{Othman2025DigitalDiplomacy} The relationship between war and social media in the modern age is a concerning issue. To understand internet censorship in Israel today, it is important to identify what individuals and institutions are behind this activity. Israel entrusts this operation to the Israeli Military Censor, a department of the Israeli Defense Forces (IDF). This group is responsible for state-sponsored censorship online and is headed by the minister of defense, currently Israel Katz.\cite{MOD_Israel} Historically, the IDF have had to answer for media censorship through their Spokesperson Unit (ISU). "the ISU is continually fluctuating between openness and opaqueness because its activities are affected by so many internal and external factors" \cite{MAGEN2018287} 

Though internet censorship can prove inflammatory, the internet can also be used to ease tensions. Digital diplomacy can be described as how a government uses the internet and related technologies to manage international relations. "Findings reveal an unmatched proactive approach by Israel’s digital diplomacy compared to other states, rooted in a humanitarian grounds concern despite limited peace efforts, and significant obstacles from prevalent anti-Israel online sentiment, changing social media perceptions, and platform executive decisions hindered by personal political inclinations." \cite{Othman2025DigitalDiplomacy} 

According to Zittrain in his 2017 paper discussing internet censorship, Israel has not always been proactive in blocking political content. "In June 2017, after a few years of no blocking, the Palestinian Authority ordered ISPs to block 12 news websites affiliated with the rival Islamist group Hamas which controls the Gaza Strip, websites affiliated with dismissed Fatah leader Mohammed Dahlan, and 10 news websites that provide news and views on Palestinian politics." \cite{zittrain2017shifting} Zittrain described this trend of blocking undesirable websites in 2017. In 2023, Israel passed what was described as "draconian" legislation by the RSF, that punishes the "consumption of terrorist materials." \cite{RSF2024IsraelCensorship} This law targeted sites such as Aljazeera, a media outlet focusing on covering the Gaza crisis funded by the Qatari government. \cite{AlJazeera2023Knesset} This example shows the litigious nature of the Israeli state in censoring content online. 

Having considered the unique national security threats faced by Israel, it is clear that citizens are not overly concerned with the State abusing its power. The tumultuous history faced by the state means that "the IDF is highly trusted by a society that deeply values the defense system, it is very difficult to criticize its deficiencies." \cite{MAGEN2018287} A troubling result of the Gaza crisis has been the utilisation of social media during war. On 14 November 2012, a tweet from the official IDF Twitter account stated “The IDF has begun a widespread campaign on terror sites \& operatives in the
Gaza Strip, chief among them Hamas \& Islamic Jihad targets." \cite{IDF_Twitter} This marked the beginning of what Kretschmer described as a war that is "tweeted live." In her research, she describes a concerning account of both sides "constantly informing on rocket attacks." \cite {Kretschmer_2017} Propaganda and misinformation has been influential in shaping global opinions on this war, and the internet has accommodated this.

