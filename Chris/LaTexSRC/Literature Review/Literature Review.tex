\chapter{Literature Review}

\section{Introduction}

Write an introduction outlining the reasons for lit review... Briefly elude to some high level differences between Ireland and Israel.

The purpose of this literature review is to survey and consider published work regarding internet censorship globally, with a particular focus on that of Ireland and Israel. Legislation, important events and other notable areas will be discussed and compared in order to gain a greater understanding the differences between internet censorship experienced in Israel and Ireland. Internet censorship is constantly evolving as it competes with privacy based tools in an 'Arms race' of sorts. This makes researching and understanding how censors achieve their purpose of particular importance. To properly understand the current situations faced by both Israeli and Irish citizens using the internet, a broad analysis of existing literature and ongoing research had to be considered. This section lays the groundwork for the thesis, detailing how both countries approach to internet censorship has evolved over the years. 

Internet usage is rising year on year globally as more users are free to surf the web. Our World in Data, an independent organisation that tracks internet usage statistics suggests that as of 2023, 67.4\% of the world was connected to the internet. \cite{owid-internet} This is a staggering number of individuals that is only set to increase. With more people relying on the internet for their livelihood, communication or otherwise, internet censorship is becoming a more pressing matter. It has also been noted previously that, based on OONI data, censorship is rising globally. This growth highlights the need for transparency and regulation surrounding user rights and privacy. 

As previously mentioned, a common misconception about the internet is that it's content is not manipulated. Another misconception held by many is that internet censorship occurs in few countries. This is also false, with censors increasing their restrictions continuously. According to Bischoff in his online article mapping internet censorship and geographies "This year we saw nearly 60 countries increase their internet censorship in some way, compared to 50 from last year’s study."\cite{bischoff2025internet} This is a troubling reality as internet content is increasingly being censored not just in authoritarian countries but by democratic states. 

\section{Literature Review Methodology}

In conducting this literature review, sources from a variety of mediums were used. Research was conducted primarily using the internet, focusing on academic papers and peer reviewed literature. Other sources like Trinity College Library, online articles and journals were also considered and sources were cross checked with relevant authorities. Information regarding country-specific legislation was taken from the official state-run website of those countries. Sources deemed potentially unreliable were placed under a higher level of scrutiny. It is worth mentioning that researching this area can be difficult as state level censorship is typically clandestine and overt. 



\section{Findings of the Literature Review}
\subsection{Ireland Historically }


\subsection{Ireland Today}





\subsection{Israel Historically}
Since the early 2000s, internet access has become increasingly available in Israel. In a paper discussing internet usage in Israel, Fisher speaks of "an increase of 152\% in the number of Israeli households connected to the Internet during the period 2000–2005." \cite{FISHER2006984} 

An important aspect of this literature review was understanding how the Israel - Palestine conflict has shaped censorship of the press and the internet over the last few decades. To better appreciate the impact of this war, the Council of Foreign Relations (CFR) provides a brief overview of notable events. \cite{CFR2024Timeline} MIFTAH, an organisation promoting open dialogue on the Israel - Palestine conflict, released a summary of freedom of press violations for the years 2000-2003. They tabulated 310 separate incidents of press freedom violations during this time, with reporters and journalists consistently being victimised. \cite{Miftah2003Censorship} It is clear from these documents that it is dangerous to report on this conflict. It is also clear that wars such as this one inevitably affect the information available to users online.

In an archived document produced by the IDF in 2016, the details for mandatory conscription of Israeli citizens is described. \cite{MOIA2016IDF} This military draft has been ongoing since 1948 when Israel declared its independence. Men are required to serve 32 months while women serve 24. This policy, in combination with Israel's renowned intelligence operation has produced highly qualified cybersecurity professionals. 

Regarding Israel's freedom of press in the 2000s, the Internet Monitor, a data analysis and collection tool states: "Modern censorship of [press] operates through voluntary agreements between the military and the Israeli Committee of Daily Newspaper Editors. Even though these agreements lack full consent from media in the country, all media organizations operating in Israel must abide by the censor's decisions." \cite{internet_monitor_israel} Though this pertains to press rather than the internet, it shows a tendency by the state to block political content. This trend would go on to continue in the 2010s and 2020s. The Colomubia Journalism Review wrote an article in 2025 discussing the potential bias of Channel 14, a prominent right-wing media outlet in Israel. Channel 14 lends itself to nationalist and patriotic rhetoric and has been subject to criticism as a result. "Netanyahu’s relationship with Channel 14 goes back years, to the time when it was Channel 20, called the Heritage Channel." \cite{CJR2024IsraelChannel14} This serves as evidence to suggest the Israeli government has a strong grasp over its media. 

\subsection{Israel Today}


Today, a large majority of Israeli citizens have access to the internet. DataReportal, a website responsible for collecting and publishing global digital reports states "there were 8.51 million internet users in Israel at the start of 2024, when internet penetration stood at 92.1 percent." \cite{Digital2024Israel} It is also pertinent to mention Israel's booming cybersecurity industry. According to YL Ventures' recent report "In 2024, the Israeli cybersecurity industry demonstrated exceptional growth," receiving \$4B in funding, double that of 2023. The roots of this industry come as a direct product of the nation's fixation with intelligence and national security. 


Reporters Without Borders (RSF), responsible for the World Press Freedom Index, provide detailed reports pertaining to media censorship globally. They have ranked Israel as 101st in the world as of 2024 in this regard. This ranking is based on the level of freedom enjoyed by journalists and media. "Press freedom is defined as the ability of journalists as individuals and collectives to select, produce, and disseminate news in the public interest independent of political, economic, legal, and social interference and in the absence of threats to their physical and mental safety." \cite{rsf_israel} 

In a 2024 paper discussing digital diplomacy in the Israel - Gaza war, Othman asserts "Governments and non-state actors leveraged social media to influence international public opinion, while misinformation campaigns complicated the narrative, undermining trust in diplomatic channels." \cite{Othman2025DigitalDiplomacy} The relationship between war and social media in the modern age is a concerning issue. To understand internet censorship in Israel today, it is important to identify what individuals and institutions are behind this activity. Israel entrusts this operation to the Israeli Military Censor, a department of the Israeli Defense Forces (IDF). This group is responsible for state-sponsored censorship online and is headed by the minister of defense, currently Israel Katz.\cite{MOD_Israel} Historically, the IDF have had to answer for media censorship through their Spokesperson Unit (ISU). "the ISU is continually fluctuating between openness and opaqueness because its activities are affected by so many internal and external factors" \cite{MAGEN2018287} 

Though internet censorship can prove inflammatory, the internet can also be used to ease tensions. Digital diplomacy can be described as how a government uses the internet and related technologies to manage international relations. "Findings reveal an unmatched proactive approach by Israel’s digital diplomacy compared to other states, rooted in a humanitarian grounds concern despite limited peace efforts, and significant obstacles from prevalent anti-Israel online sentiment, changing social media perceptions, and platform executive decisions hindered by personal political inclinations." \cite{Othman2025DigitalDiplomacy} 

According to Zittrain in his 2017 paper discussing internet censorship, Israel has not always been proactive in blocking political content. "In June 2017, after a few years of no blocking, the Palestinian Authority ordered ISPs to block 12 news websites affiliated with the rival Islamist group Hamas which controls the Gaza Strip, websites affiliated with dismissed Fatah leader Mohammed Dahlan, and 10 news websites that provide news and views on Palestinian politics." \cite{zittrain2017shifting} Zittrain described this trend of blocking undesirable websites in 2017. In 2023, Israel passed what was described as "draconian" legislation by the RSF, that punishes the "consumption of terrorist materials." \cite{RSF2024IsraelCensorship} This law targeted sites such as Aljazeera, a media outlet focusing on covering the Gaza crisis funded by the Qatari government. \cite{AlJazeera2023Knesset} This example shows the litigious nature of the Israeli state in censoring content online. 

Having considered the unique national security threats faced by Israel, it is clear that citizens are not overly concerned with the State abusing its power. The tumultuous history faced by the state means that "the IDF is highly trusted by a society that deeply values the defense system, it is very difficult to criticize its deficiencies." \cite{MAGEN2018287} A troubling result of the Gaza crisis has been the utilisation of social media during war. On 14 November 2012, a tweet from the official IDF Twitter account stated “The IDF has begun a widespread campaign on terror sites \& operatives in the
#Gaza Strip, chief among them #Hamas \& Islamic Jihad targets." \cite{IDF_Twitter} This marked the beginning of what Kretschmer described as a war that is "tweeted live." In her research, she describes a concerning account of both sides "constantly informing on rocket attacks." \cite {Kretschmer_2017} Propaganda and misinformation has been influential in shaping global opinions on this war, and the internet has accommodated this.

\section{Analysis: Ireland vs Israel}






\section{Conclusions}

